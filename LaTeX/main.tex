\documentclass{article}
\usepackage{float}
% Language setting
% Replace `english' with e.g. `spanish' to change the document language
\usepackage[english]{babel}

% Set page size and margins
% Replace `letterpaper' with `a4paper' for UK/EU standard size
\usepackage[a4paper,top=2cm,bottom=2cm,left=3cm,right=3cm,marginparwidth=1.75cm]{geometry}

% Useful packages
\usepackage{amsmath}
\usepackage{graphicx}
\usepackage[colorlinks=true, allcolors=blue]{hyperref}
\usepackage{hyperref}

%User defined packages
\usepackage[table]{xcolor}

\setlength{\arrayrulewidth}{1mm}
\setlength{\tabcolsep}{18pt}
\renewcommand{\arraystretch}{2.5}
\newcolumntype{s}{>{\columncolor[HTML]{dbf7fa}} p{6cm}}
\arrayrulecolor[HTML]{3352FF}

\title{RobonAUT autó megvalósítási terve}
\author{Nevezi Virág \\
      \and
      Szanda Márton\\
      \and
      Varga Dávid
      }

\begin{document}
\maketitle
\newpage

\section{A kiindulás}
\subsection{Ötletek}

\centering
\begin{tabular}{s|s}
Alkatrészek & Specifikációk\\\hline
Főtáp & 42 \\
Segédtáp & 13 \\
DC motor & 13 \\
Rádió vevő (Gyári) & 13 \\
Rádió vevő (Silabs) & 13 \\
Távirányító & 13 \\
Vonalszenzor & 13 \\
Szervó & \href{https://www.srt-rc.com/index.php?id=187}{SRT BH922S}\\
Infravörös távolság szenzor & \href{https://www.sparkfun.com/datasheets/Sensors/Infrared/gp2y0a02yk_e.pdf}{SHARP GP2Y0A02YK0F}\\
Mágneses inkrementális adó & \href{https://www.alldatasheet.com/html-pdf/1148492/ALLEGRO/A1333LLETR-5-T/349/1/A1333LLETR-5-T.html}{ALLEGRO A1333LLETR-T}\\
Inerciális szenzor & LSM6DSO\\
\end{tabular}

Kezdeti tervek:

\begin{itemize}
\item Nem szeretnénk hátsó kormányzást
\item Eltolt PWM
\item Segédtáp
\item Akksi
\item Deadman switch
\item Távolságkövetés miatt 2 szenzor (szenzor látószöge, mert kanyarban baj lehet)
\item 1,5 m/s max
\item Hogyan érdemes haladni?
\end{itemize}

\subsection{Első meeting (09.24.)}

Virág jegyzete:

Prio: Q1-re a három szabályozási körnek biztosnak kell lennie

\vspace{5mm} %5mm vertical space

Kompetenciákat fel kell osztani:
\begin{itemize}
\item Dávid - szabályozási kör: beavatkozás meg mérés, 
\begin{itemize}
    \item 2 db PWM -> leprogramozni, kimenetet oszcilloszkóppal mérni, eltolt módszer; szervókat be kell hangolni, 1,5 ms-osat ráadni és utána pici lépésenként beállítani a középállást meg a végállásokat, minden szétszedés és összerakás után meg kell csinálni; figyelni a cutoffra; impulzus hosszba kódolt rcp alapjel, ez egy üzenetformátum, NEM PWM1
\end{itemize}
\item Marci - státuszinformációk vétele az autóról -> kliensoldali alkalmazás hozzá (pl: Python,.json,.yaml (FONTOS)); 
\begin{itemize}
    \item parancsértelmező alkalmazás írása STM32-re, 
    \item paraméterek állítása c-ben (FONTOS); mikrokontroller: \item szenzorokból adatokat kinyerni, 
    \item amíg nincs NYÁK a programozáshoz, nucleora rákötni őket, 
    \item kinyerni a  nyers adatokat, 
    \item perifériák élesztése
\end{itemize}

\item Virág - pontos legyen, hamar kell, 
\begin{itemize}
    \item PCB tervek: 
    \item2 db vonalszenzor, 
    \item4db táv szenzor,(megtervezni nem kell), 
    \item hátsó kerék kormányzás meggondolandó, 
    \item motor meghajtó kártya (megtervezni nem kell), 
    \item központi modul, amire a nucleo, 
    \item rádiós modul, vonalszenzorok bekötése meg, kommunikáció: vagy a raspberrys (figyelembe menni választani) debuggeres cucc vagy  
    \item pedig a legegyszerűbb megoldás (Bluethoot 3.0 chip) -> pros vs cons, rákeresni, 
    \item kijelző vagy státusz LEDek, gombok -> lehet külön kisebb nyákra kéne?, 
    \item hangszóró (+alkatrészek), 
    \item egyensúlyozáshoz forgó elem részei, 
    \item bekötni dead man switchnek a távirányító vevőjét (szervókábel csatlakozók hozzá), 
    \item tápok: NIMT hibrid + 3 cellás LiPo -> megnézni, hogy mi hány V-ot vár, tápmodulok (?), 2-3 szervónak helyet hagyni, LÁBKIOSZTÁS (processzor periféria készletét figyelembe venni)
\end{itemize}
\item Következő hétre: kapcsolási terv; azb összes perifériát megbeszélni
\item Második hét: elrendezési terv
\item Harmadik hét: huzalozási terv
\end{itemize}

Dávid meeting felírása:

\vspace{5mm} %5mm vertical space

Napirend:

Találjuk ki, hogy milyen alkatrészek kellenek!
Kiegészítés: LCD display, 4 IR szenzor, 2 szervó(gyári és srt), passzív áramköri elemek (táp modulok) mondjuk feszültségillesztéshez, egy extra nyák amit elérhető helyre teszünk debug/reset funkciókkal. Bluetooth modul a kommunikációhoz, szabályzó paraméter állításhoz.

\vspace{5mm} %5mm vertical space

FELADATOK:
\begin{itemize}
\item Kapcsolási rajz
\item Proci periféria készletét és lábkiosztását meg kell nézni és kiosztani
\item LUT megírása IR szenzorhoz
\item UART kommunikáció élesztése, parancsok fogadása->ledvillogtatás pl
\item client megírása vonalérzékelő adatainak, állapotoknak, erroroknak a megjelenítéséhez.
\item Motorszabályzó PWM eltolt vezerles, oszcilloszkópos mérés
\end{itemize}

!!! FONTOS !!!

\vspace{5mm} %5mm vertical space

Logó kérdés eldöntése péntekig (10.04.)
Igazából az alap ötletem az volt, hogy lehetne egy Q1 logó digital clock stílusban, ahogy az a \ref{fig:logo_v0} képen is látszik. Kicsit üresnek éreztem, ezért gondoltam, hogy két kart raknék az oldalára, mint ahogy a Teremtés festményen is van. Az egyik egy robotkar lenne, a másik meg emberi. Az egész stilizált, letisztult lenne, fekete-fehér, a lentebb látható kép csak referencia a logó elkészítéséhez.

\newpage
\begin{figure}
      \centering
      \includegraphics[width=0.5\linewidth]{logo_v0.png}
      \caption{\label{fig:logo_v0}Logó ötlet}
      \end{figure}

\begin{figure}
\centering
\includegraphics[width=0.25\linewidth]{korondiP.jpg}
\caption{\label{fig:korondi}This Korondi was uploaded via Debreceni Egyetem.}
\end{figure}

\subsection{Infravörös távolság szenzor lookup table}
\begin{figure}[H]
    \centering
    \includegraphics[scale=0.7]{IR_distance_characteristics.png}
    \caption{\label{fig:korondi}IR sensor characteristics}
\end{figure}
A görbét lineáris interpolációval közelítve pyton-ban:

\begin{figure}[H]
    \centering
    \includegraphics[scale=0.7]{Interp.png}
    \caption{\label{fig:korondi}IR sensor characteristics}
\end{figure}



\subsection{How to add Citations and a References List}

You can simply upload a \verb|.bib| file containing your BibTeX entries, created with a tool such as JabRef. You can then cite entries from it, like this: \cite{greenwade93}. Just remember to specify a bibliography style, as well as the filename of the \verb|.bib|. You can find a \href{https://www.overleaf.com/help/97-how-to-include-a-bibliography-using-bibtex}{video tutorial here} to learn more about BibTeX.

If you have an \href{https://www.overleaf.com/user/subscription/plans}{upgraded account}, you can also import your Mendeley or Zotero library directly as a \verb|.bib| file, via the upload menu in the file-tree.

\subsection{Good luck!}

We hope you find Overleaf useful, and do take a look at our \href{https://www.overleaf.com/learn}{help library} for more tutorials and user guides! Please also let us know if you have any feedback using the Contact Us link at the bottom of the Overleaf menu --- or use the contact form at \url{https://www.overleaf.com/contact}.

\bibliographystyle{alpha}
\bibliography{sample}

\end{document}